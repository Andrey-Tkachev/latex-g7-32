\sloppy 

% 1. Настройки стиля ГОСТ 7-32
% Для начала определяем, хотим мы или нет, чтобы рисунки и таблицы нумеровались в пределах раздела, или нам нужна сквозная нумерация.
% А не забыл ли автор букву 't' ?
\EqInChapter % формулы будут нумероваться в пределах раздела
\TableInChapter % таблицы будут нумероваться в пределах раздела
\PicInChapter % рисунки будут нумероваться в пределах раздела

% 2. Добавляем гипертекстовое оглавление в PDF
\usepackage[
bookmarks=true, colorlinks=true, unicode=true,
urlcolor=black,linkcolor=black, anchorcolor=black,
citecolor=black, menucolor=black, filecolor=bla,
]{hyperref}

% 3. Изменение начертания шрифта --- после чего выглядит таймсоподобно.
\newcommand{\Font}{\usefont{T2A}{ftm}{m}{}}
% Шрифт в подписях к рисункам.
\renewcommand{\captionfont}{\Font}
\renewcommand{\captionlabelfont}{\Font}
% Шрифт в номерах страниц
\renewcommand{\thepage}{\Font\arabic{page}}
% Шрифт в номерах формул (предполагается \EqInChaper) 
\renewcommand{\theequation}{\Font\thechapter.\arabic{equation}} 

% 4. Прочие полезные пакеты.
\usepackage{cmap} % теперь из pdf можно копипастить русский текст
\usepackage{underscore} % Ура! Теперь можно писать подчёркивание.
\usepackage{graphicx}   % Пакет для включения рисунков
 
 % 5. Любимые команды
\newcommand{\Code}[1]{\textbf{#1}}
