%% Преамбула TeX-файла

% 1. Стиль и язык
\documentclass[utf8]{G7-32} % Стиль (по умолчанию будет 14pt)

% Остальные стандартные настройки убраны в preamble-std.tex
\sloppy 

% 1. Настройки стиля ГОСТ 7-32
% Для начала определяем, хотим мы или нет, чтобы рисунки и таблицы нумеровались в пределах раздела, или нам нужна сквозная нумерация.
% А не забыл ли автор букву 't' ?
\EqInChapter % формулы будут нумероваться в пределах раздела
\TableInChapter % таблицы будут нумероваться в пределах раздела
\PicInChapter % рисунки будут нумероваться в пределах раздела

% 2. Добавляем гипертекстовое оглавление в PDF
\usepackage[
bookmarks=true, colorlinks=true, unicode=true,
urlcolor=black,linkcolor=black, anchorcolor=black,
citecolor=black, menucolor=black, filecolor=bla,
]{hyperref}

% 3. Изменение начертания шрифта --- после чего выглядит таймсоподобно.
\usepackage{cyrtimes} % apt-get install scalable-cyrfonts-tex

% 4. Прочие полезные пакеты.
\usepackage{cmap} % теперь из pdf можно копипастить русский текст
\usepackage{underscore} % Ура! Теперь можно писать подчёркивание.
\usepackage{graphicx}   % Пакет для включения рисунков
 
 % 5. Любимые команды
\newcommand{\Code}[1]{\textbf{#1}}


% С такими оно полями оно работает по-умолчанию:
% \RequirePackage[left=20mm,right=10mm,top=20mm,bottom=20mm,headsep=0pt]{geometry}
% Если вас тошнит от поля в 10мм --- увеличивайте до 20-ти, ну и про переплёт не забывайте:
\geometry{right=20mm}
\geometry{left=30mm}

\begin{document} 

\frontmatter % выключает нумерацию ВСЕГО; здесь начинаются ненумерованные главы: реферат, введение, глоссарий, сокращения и прочее

% Также можно использовать \Referat, как в оригинале
\begin{abstract}
Это пример каркаса расчётно-пояснительной записки, желательный к использованию в РПЗ проекта по курсу РСОИ.

Дополняет краткое пособие по графике в Latex.  Данный опус, как и более новые версии этого документа, можно взять по адресу (\url{http://sevik.ru/latex}). Минимально необходимые пакеты Latex, которые должны стоять: mathtext, amssymb, amsmath, icomma, longtable, graphicx, underscore, cmap, hyperref.

Текст в документе носит совершенно абстрактный характер.
\end{abstract}

\tableofcontents

% \Defines % Необходимые определения. Вряд ли понадобться
% \begin{description}
% \item[Распределённый] Слово, которое нельзя употреблять.
%  \end{description}

\Abbreviations %% Список обозначений и сокращений в тексте
\begin{description}
\item[АИС] Автоматизированная информационная система.
\end{description}

\Introduction

Целью работы является создание всякой всячины. Для достижения поставленной цели необходимо решить следующие задачи:
%
\begin{itemize}
\item проанализировать существующую всячину;
\item спроектировать свою, новую всячину;
\item изготовить всякую всячину;
\item проверить её работоспособность.
\end{itemize}

\mainmatter % это включает нумерацию глав и секций в документе ниже

\chapter{Аналитический раздел}
%
% % В начале раздела  можно напмнить его цель
%
В данном разделе анализируется и классифицируется существующая всячина и пути создания новой всячины. А вот отступ справа в 1 см.~--- это хоть и по ГОСТ, но ведь диагноз же...

\section{Анализ того и сего}

\textbf{textbf}

\begin{figure}
\centering
\caption{Test textbf (\textbf{textbf}).}
\label{fig:dia}
\end{figure}


В \cite{Pup99} указано, что...

Кстати, про картинки. Во-первых, для фигур следует использовать \texttt{[ht]}. Если и после этого картинки вставляются <<не по ГОСТ>>, т.е. слишком далеко от места ссылки,~--- значит у вас в РПЗ \textbf{слишком мало текста}! Хотя и ужасный параметр \texttt{!ht} у окружения \texttt{figure} тоже никто не отменял, только при его использовании документ получается страшный, как в ворде, поэтому просьба так не делать по возможности.


\section{Существующие подходы к созданию всячины}

Известны следующие подходы...

\begin{enumerate}
\item Перечисление с номерами.
\item Номера первого уровня. Да, ГОСТ требует именно так~--- сначала буквы, на втором уровне~--- цифры. Увы.
\begin{enumerate} 
\item Номера второго уровня.
\item Номера второго уровня.
\end{enumerate}
\end{enumerate}

В отчётах могут быть и таблицы - см. табл.~\ref{tab:tabular} и~\ref{tab:longtable}.

Небольшая таблица делается при помощи \Code{tabular} внутри \Code{table} (последний
полностью аналогичен \Code{figure}, но добавляет другую подпись).

\begin{table}[ht]
  \caption{Пример короткой таблицы с длинным названием на много длинных-длинных строк}
  \begin{tabular}{|r|c|c|c|l|}
  \hline
  Тело      & $F$ & $V$  & $E$ & $F+V-E-2$ \\
  \hline                       
  Тетраэдр  & 4   & 4    & 6   & 0         \\ 
  Куб       & 6   & 8    & 12  & 0         \\ 
  Октаэдр   & 8   & 6    & 12  & 0         \\ 
  Додекаэдр & 20  & 12   & 30  & 0         \\ 
  Икосаэдр  & 12  & 20   & 30  & 0         \\ 
  \hline
  Эйлер     & 666 & 9000 & 42  & $+\infty$ \\
  \hline
  \end{tabular}      
  \label{tab:tabular}
\end{table}

Для больших таблиц следует использовать пакет \Code{longtable}, позволяющий создавать
таблицы на несколько страниц по ГОСТ.

Для того, чтобы длинный текст разбивался на много строк в пределах одной ячейки, надо в
качестве ее формата задавать \texttt{p} и указывать явно ширину: в мм/дюймах
(\texttt{110mm}), относительно ширины страницы (\texttt{0.22\textbackslash textwidth})
и~т.п.

Можно также использовать уменьшенный шрифт -- но, пожалуйста, тогда уж во \textbf{всей}
таблице сразу.

\begin{center}
  \begin{longtable}{|p{0.40\textwidth}|c|p{0.30\textwidth}|}
    \caption{Пример длинной таблицы с длинным названием на много длинных-длинных строк}
    \label{tab:longtable}
    \hline
    Вид шума & Громкость, дБ & Комментарий \\
    \hline \endfirsthead
    \subcaption{Продолжение таблицы~\ref{tab:longtable}}
    \hline \endhead
    \hline \subcaption{Продолжение на след. стр.}
    \endfoot
    \hline \endlastfoot
    Порог слышимости             & 0     &                                                \\
    \hline
    Шепот в тихой библиотеке     & 30    &                                                \\
    Обычный разговор             & 60-70 &                                                \\
    Звонок телефона              & 80    & \small{Конечно, это было до эпохи мобильников} \\
    Уличный шум                  & 85    & \small{(внутри машины)}                        \\
    Гудок поезда                 & 90    &                                                \\
    Шум электрички               & 95    &                                                \\
    \hline
    Порог здоровой нормы         & 90-95 & \small{Длительное пребывание на более
    громком шуме может привести к ухудшению слуха}                                        \\
    \hline
    Мотоцикл                     & 100   &                                                \\
    Power Mower                  & 107   & \small{(модель бензокосилки)}                  \\
    Бензопила                    & 110   & \small{(Doom в целом вреден для здоровья)}     \\
    Рок-концерт                  & 115   &                                                \\
    \hline
    Порог боли                   & 125   & \small{feel the pain}                          \\
    \hline
    Клепальный молоток           & 125   & \small{(автор сам не знает, что это)}          \\
    \hline
    Порог опасности              & 140   & \small{Даже кратковременное пребывание на
    шуме большего уровня может привести к необратимым последствиям}                       \\
    \hline
    Реактивный двигатель         & 140   &                                                \\
                                 & 180   & \small{Необратимое полное повреждение
                                 слуховых органов}                                        \\
    Самый громкий возможный звук & 194   & \small{Интересно, почему?..}                   \\
  \end{longtable}
\end{center}

\chapter{Конструкторский раздел}

В данном разделе проектируется новая всячина.

\section{Архитектура всячины}

\paragraph{Проверка} параграфа. Вроде работает.
\paragraph{Вторая проверка} параграфа. Опять работает.

Вот.

\begin{itemize}
\item Это список с <<палочками>>.
\item Хотя он и не по ГОСТ, кажется.
\end{itemize}

\begin{enumerate}
\item Поэтому для списка, начинающегося с заглавной буквы, лучше список с цифрами.
\end{enumerate}

Формула \ref{F:F1} совершено бессмысленна.

%Кстати, при каких-то условиях <<удавалось>> получить двойный скобки вокруг номеров формул. Вопрос исследуется.

\begin{equation}
a= cb
\label{F:F1}
\end{equation}


Окружение \texttt{cases} опять работает (см. \ref{F:F2}), спасибо И. Короткову за исправления..


\begin{equation}
a= \begin{cases}
 3x + 5y + z, \mbox{если хорошо} \\
 7x - 2y + 4z, \mbox{если плохо}\\
 -6x + 3y + 2z, \mbox{если совсем плохо}
\end{cases}
\label{F:F2}
\end{equation}

\section{Подсистема всякой ерунды}

\subsection{Блок-схема всякой ерунды}


\subsubsection*{Кстати о заголовках}

У нас есть и \Code{subsubsection}. Только лучше её не нумеровать.

\chapter{Технологический раздел}



В данном разделе описано изготовление и требование всячины. Благодаря пакет \Code{underscore} эскейпить подчёркивание  не нужно (\Code{some_function}).

Если нам нужно засунуть в latex какой-то код, то можно использовать окружение \Code{verbatim}. Только следует помнить, что табы в нём <<съедаются>>. 
\begin{verbatim}
a_b = a + b; // русский комментарий
if (a_b > 0)
    a_b = 0;
\end{verbatim}

Ещё для этих целей есть пакет \Code{listings}. К сожаление, пакет \Code{listings} всё ещё работает криво при появлении в листинге русских букв и кодировке исходников utf-8. 

\chapter{Исследовательский раздел}

В данном разделе исследуется работоспособность всячины.

\backmatter %% Здесь заканчивается нумерованная часть документа и начинаются ссылки и заключение

\Conclusion % заключение к отчёту

В результате проделанной работы стало ясно, что ничего не ясно...
%
% % Список литературы.
\begin{thebibliography}{99}
\bibitem{Pup99} Василий Пупкин. Latex для <<чайников>>.~--- М.: 2009~--- 50 c.
\end{thebibliography}

\end{document}


