\chapter{Технологический раздел}
\label{cha:impl}

В данном разделе описано изготовление и требование всячины. Благодаря пакет \Code{underscore} эскейпить подчёркивание  не нужно (\Code{some_function}).

Для вставки кода есть пакет \Code{listings}. К сожалению, пакет \Code{listings} всё ещё
работает криво при появлении в листинге русских букв и кодировке исходников utf-8. Есть
альтернатива \Code{listingsutf8}, однако она работает лишь с
\texttt{\textbackslash lstinputlisting}, но не с окружением \Code{lstlisting}

Вот так можно вставлять псевдокод (питоноподобный язык определен в шаблоне):

\begin{lstlisting}[style=pseudocode,caption={Алгоритм оценки дипломных работ}]
def EvaluateDiplomas():
    for each student in Masters:
        student.Mark := 5
    for each student in Engineers:
        if Good(student):
            student.Mark := 5
        else:
            student.Mark := 4
\end{lstlisting}

Еще в шаблоне определен псевдоязык для BNF:

\begin{lstlisting}[style=grammar,basicstyle=\small,caption={Грамматика}]
  ifstmt -> "if" "(" expression ")" stmt |
            "if" "(" expression ")" stmt1 "else" stmt2
  number -> digit digit*
\end{lstlisting}

В листинге~\ref{lst:sample01} работают русские буквы. Сильная магия.

\lstinputlisting[language=C,caption=Пример (\Code{test.c}),label=lst:sample01]{../src/test.c}

% Для вставки реального кода лучше использовать \texttt{\textbackslash lstinputlisting} (который понимает
% UTF8) и стили \Code{realcode} либо \Code{simplecode} (в зависимости от размера куска).




Можно также использовать окружение \Code{verbatim}, если \Code{listings} чем-то не
устраивает. Только следует помнить, что табы в нём <<съедаются>>. Существует так же команда \Code{verbatiminput} для вставки файла.

\begin{verbatim}
a_b = a + b; // русский комментарий
if (a_b > 0)
    a_b = 0;
\end{verbatim}

%%% Local Variables: 
%%% mode: latex
%%% TeX-master: "rpz"
%%% End: 
