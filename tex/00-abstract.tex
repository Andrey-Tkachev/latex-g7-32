% Также можно использовать \Referat, как в оригинале
\begin{abstract}

    Отчет содержит \pageref{LastPage}\,стр.
    \ifnum \totfig >0
    , \totfig~рис.%
\fi
\ifnum \tottab >0
    , \tottab~табл.%
\fi
%
\ifnum \totbib =1
    , \totbib~источник%
\else
    \ifnum \totbib >1
        \ifnum \totbib <5
            , \totbib~источника%
        \else
            , \totbib~источников%
        \fi
    \fi
\fi
%
\ifnum \totapp >0
    , \totapp~прил.%
\fi


Это пример каркаса расчётно-пояснительной записки, желательный к использованию в РПЗ проекта по курсу РСОИ
\nocite{*}.

Данный опус, как и более новые версии этого документа, можно взять по адресу (\url{https://github.com/rominf/latex-g7-32}).

Текст в документе носит совершенно абстрактный характер.

СПИСОК!!!

\begin{itemize}
    \item Элемент 0 уровня 
    \item Ещё элемент 0 уровня 
        \begin{itemize}
            \item 1 уровень 
            \item Ещё элемент 
            \item И ещё один 1 уровня элемент и ещё один 1 уровня элемент и ещё один 1 уровня элемент и ещё один 1 уровня элемент и ещё один 1 уровня элемент и ещё один 1 уровня элемент и ещё один 1 уровня элемент и ещё один 1 уровня элемент и ещё один 1 уровня элемент и ещё один 1 уровня элемент
                \begin{itemize}
                    \item Элемент 2 уровня  
                    \item Ещё элемент 2 уровня Ещё элемент 2 уровня Ещё элемент 2 уровня Ещё элемент 2 уровня Ещё элемент 2 уровня Ещё элемент 2 уровня Ещё элемент 2 уровня Ещё элемент 2 уровня Ещё элемент 2 уровня Ещё элемент 2 уровня
                        \begin{itemize}
                            \item Элемент 3 уровня  
                            \item Ещё элемент 3 уровня Ещё элемент 2 уровня Ещё элемент 2 уровня Ещё элемент 3 уровня Ещё элемент 2 уровня Ещё элемент 2 уровня Ещё элемент 3 уровня Ещё элемент 2 уровня Ещё элемент 2 уровня Ещё элемент 3 уровня
                        \end{itemize}
                \end{itemize}
        \end{itemize}
\end{itemize}


А вот тут список кончился, это просто текст.  		А вот тут список кончился, это просто текст. 		А вот тут список кончился, это просто текст. 		А вот тут список кончился, это просто текст. 		А вот тут список кончился, это просто текст. 		А вот тут список кончился, это просто текст.

И ещё один список, уже нумерованный!

\begin{enumerate}
    \item Входные порты также обладают переключателем мультиплексности (\texttt{multiport}), т.е. позволяют создавать множественные входные подключения. 

        В этом случае значение такого порта всегда является списком (например, нулевой длины).
    \item Входные порты также обладают переключателем мультиплексности (\texttt{multiport}), т.е. позволяют создавать множественные входные подключения. В этом случае значение такого порта всегда является списком (например, нулевой длины).
        \begin{enumerate}
            \item Входные порты также обладают переключателем мультиплексности (\texttt{multiport}), т.е. позволяют создавать множественные входные подключения. 

                В этом случае значение такого порта всегда является списком (например, нулевой длины).
                \begin{enumerate}
                    \item Входные порты также обладают переключателем мультиплексности (\texttt{multiport}), т.е. позволяют создавать множественные входные подключения. 

                        В этом случае значение такого порта всегда является списком (например, нулевой длины).

                \end{enumerate}
        \end{enumerate}
    \item Элемент 0 уровня 
    \item Элемент 0 уровня 
    \item Элемент 0 уровня 
    \item Элемент 0 уровня 
    \item Элемент 0 уровня 
    \item Элемент 0 уровня 
    \item Элемент 0 уровня 
    \item Элемент 0 уровня 
    \item Элемент 0 уровня 
    \item Элемент 0 уровня 
    \item Элемент 0 уровня 
    \item Элемент 0 уровня 
    \item Элемент 0 уровня 
    \item Элемент 0 уровня 
    \item Элемент 0 уровня 
    \item Элемент 0 уровня 
    \item Элемент 0 уровня 
    \item Элемент 0 уровня 
    \item Элемент 0 уровня 
    \item Элемент 0 уровня 
    \item Элемент 0 уровня 
    \item Элемент 0 уровня 
    \item Элемент 0 уровня 
\end{enumerate}



\end{abstract}

%%% Local Variables: 
%%% mode: latex
%%% TeX-master: "rpz"
%%% End: 
