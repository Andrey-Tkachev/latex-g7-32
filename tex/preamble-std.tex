\sloppy 

% 1. Настройки стиля ГОСТ 7-32
% Для начала определяем, хотим мы или нет, чтобы рисунки и таблицы нумеровались в пределах раздела, или нам нужна сквозная нумерация.
% А не забыл ли автор букву 't' ?
\EqInChapter % формулы будут нумероваться в пределах раздела
\TableInChapter % таблицы будут нумероваться в пределах раздела
\PicInChapter % рисунки будут нумероваться в пределах раздела

% 2. Добавляем гипертекстовое оглавление в PDF
\usepackage[
bookmarks=true, colorlinks=true, unicode=true,
urlcolor=black,linkcolor=black, anchorcolor=black,
citecolor=black, menucolor=black, filecolor=black,
]{hyperref}

% 3. Изменение начертания шрифта --- после чего выглядит таймсоподобно.
% apt-get install scalable-cyrfonts-tex

\IfFileExists{cyrtimes.sty}
    {
        \usepackage{cyrtimespatched}
    } 
    {
        % А если Times нету, то будет CM...
    }


% 4. Прочие полезные пакеты.
\usepackage{underscore} % Ура! Теперь можно писать подчёркивание. 
                        % И нельзя использовать подчёркивание в файлах. 
                        % Выбирай, но осторожно.

\usepackage{graphicx}   % Пакет для включения рисунков
 
 % 5. Любимые команды
\newcommand{\Code}[1]{\textbf{#1}}

% 6. Поля
% С такими оно полями оно работает по-умолчанию:
% \RequirePackage[left=20mm,right=10mm,top=20mm,bottom=20mm,headsep=0pt]{geometry}
% Если вас тошнит от поля в 10мм --- увеличивайте до 20-ти, ну и про переплёт не забывайте:
\geometry{right=20mm}
\geometry{left=30mm}
